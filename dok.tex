% TeX encoding = utf8
% TeX spellcheck = pl_PL 
\documentclass[a4paper, 11pt]{article}
\usepackage[utf8]{inputenc}
\usepackage[polish]{babel}
\usepackage{polski}
\usepackage{float}
\usepackage{graphicx}
\usepackage{listings}
\usepackage{amsfonts}
\usepackage{geometry}
\usepackage{multicol}
\usepackage{latexsym}
\usepackage{enumerate}
\usepackage{hyperref}
\usepackage{wrapfig}
\usepackage{color} %red, green, blue, yellow, cyan, magenta, black, white
\definecolor{mygreen}{RGB}{28,172,0} % color values Red, Green, Blue
\definecolor{mylilas}{RGB}{170,55,241}

\author{Kamil Foryszewski}
\title{Sterowanie Procesami - projekt I, zadanie 1.33}
\frenchspacing

\newgeometry{tmargin=2cm, bmargin=2cm, lmargin=2cm, rmargin=2cm}
\pagestyle{empty}


\begin{document}

\lstset{language=Matlab,%
    basicstyle=\color{red},
    breaklines=true,%
    morekeywords={matlab2tikz},
    keywordstyle=\color{blue},%
    morekeywords=[2]{1}, keywordstyle=[2]{\color{black}},
    identifierstyle=\color{black},%
    stringstyle=\color{mylilas},%
    commentstyle=\color{mygreen},%
    showstringspaces=false,
    numbers=right,%
    numberstyle={ \color{black}},% size of the numbers
    numbersep=5pt, % this defines how far the numbers are from the text
    emph=[1]{for,endfor,endwhile,endfunction,endif,break},emphstyle=[1]\color{blue}, %some words to emphasise
    emph=[2]{,.}, emphstyle=[2]\color{yellow},%
}

\maketitle
\tableofcontents

\section{Polecenie 1}

\subsection{Ciągły obiekt dynamiczny}
Dany jest ciągły obiekt dynamiczny o transmitancji: 
$$G(s) = \frac{2s^2+22s+48}{s^3+8s^2-65s-504} = \frac{2(s+8)(s+3)}{(s+7)(s-8)(s+9)}$$

\subsection{Transmitancja dyskretna}
Wyznaczona transmitancja dyskretna ma następującą postać: 
$$G(s) = \frac{2s^2+22s+48}{s^3+8s^2-65s-504} = \frac{2(s+8)(s+3)}{(s+7)(s-8)(s+9)}$$
Została wyznaczona przy pomocy polecenia \emph{c2d} pakietu matlab. 
$$Gs = tf([2\quad 22\quad 48], [1\quad 8\quad -65\quad -504])$$
Na początku został utworzony model transmitancji ciągłej wykorzystany poźniej do wyznaczenia transmitancji dyskretnej. 
$$Gz = c2d(G,0.1,\ 'zoh')$$
Gdzie $0.1$ to okres próbkowania, \emph{'zoh'} oznacza ekstrapolator zerowego rzędu. 
\subsection{Zera i bieguny transmitancji}
Dla transmitancji ciągłej zera i bieguny odczytujemy bezpośrednio ze wzoru:\\
Zera: \\
$s_0^1 = -8$\\
$s_0^2 = 3$\\
Bieguny: \\
$s_b^1 = -7$\\
$s_b^2 = 8$\\
$s_b^3 = -9$\\
\\
Dla transmitancji dyskretnej zera i bieguny zostały wyznaczone funkcją \emph{root} pakietu matlab: 







\end{document}