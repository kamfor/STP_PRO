% TeX encoding = utf8
% TeX spellcheck = pl_PL 
\documentclass[a4paper, 11pt]{article}
\usepackage[utf8]{inputenc}
\usepackage[polish]{babel}
\usepackage{polski}
\usepackage{float}
\usepackage{graphicx}
\usepackage{listings}
\usepackage{amsfonts}
\usepackage{geometry}
\usepackage{multicol}
\usepackage{latexsym}
\usepackage{enumerate}
\usepackage{hyperref}
\usepackage{wrapfig}
\usepackage{color} %red, green, blue, yellow, cyan, magenta, black, white
\definecolor{mygreen}{RGB}{28,172,0} % color values Red, Green, Blue
\definecolor{mylilas}{RGB}{170,55,241}

\author{Kamil Foryszewski}
\title{Sterowanie Procesami - projekt 2, zadanie 2.16}
\frenchspacing

\newgeometry{tmargin=2cm, bmargin=2cm, lmargin=2cm, rmargin=2cm}
\pagestyle{empty}


\begin{document}

\lstset{language=Matlab,%
    basicstyle=\color{red},
    breaklines=true,%
    morekeywords={matlab2tikz},
    keywordstyle=\color{blue},%
    morekeywords=[2]{1}, keywordstyle=[2]{\color{black}},
    identifierstyle=\color{black},%
    stringstyle=\color{mylilas},%
    commentstyle=\color{mygreen},%
    showstringspaces=false,
    numbers=right,%
    numberstyle={ \color{black}},% size of the numbers
    numbersep=5pt, % this defines how far the numbers are from the text
    emph=[1]{for,endfor,endwhile,endfunction,endif,break},emphstyle=[1]\color{blue}, %some words to emphasise
    emph=[2]{,.}, emphstyle=[2]\color{yellow},%
}

\maketitle
\tableofcontents

\section{Polecenie}
Obiekt regulacji jest opisany transmitancją: 
$$G(s) = \frac{K_0e^{-T_0s}}{(T_1 + 1)(T_2+1)}  = \frac{5e^{-T_0s}}{10.04s^2+7.02s +1}  $$
Gdzie: 
$K_0 = 3.5, T_0 = 5, T_1 = 2, T_2 = 5.02$

\section{Transmitancja dyskretna}
Wyznaczona transmitancja dyskretna ma następującą postać: 
$$G(z) = z^{-10}\frac{0.0388z+0.0346}{z^2-1.684z+0.705}$$
Została wyznaczona przy pomocy polecenia \emph{c2d} pakietu matlab. 
\begin{figure}[htp]
\centering
\includegraphics[scale=0.6]{1_1.png}
\caption{Odpowiedź skokowa transmitancji ciągłej i dyskretnej}
\label{}
\end{figure}%tutaj rysuneczek z matlaba 
Odpowiedź skokowa obu transmitancji w przybliżeniu jest taka sama. Odpowiedzi skokowe obliczone jako granice transmitancji przy argumentach $s$dążącym do $0$ i $z$ dażącym do $1$ są również bardzo zbliżone. Wzmocnienie statyczne transmitancji ciągłej $K_c = 5$, natomiast wzmocnienie statyczne transmitancji dyskretnej $K_d = 5.0572$
\section{Równanie różnicowe}
Korzystając z transmitancji możemy wyznaczyć równanie różnicowe opisujące obiekt w postaci: 

$$y(k) = \sum_{i=1}^{n} b_iy(k-i)+ \sum_{i=1}^{m} c_iu(k-1)$$
Należy zatem przekształcić transmitancję do następującej postaci: 
$$G(z) = \frac{0.0388z^{-11}+0.0346z^{-12}}{z-1.684z^{-1}+0.705z^{-2}} = \frac{Y(z)}{U(z)}$$
Po przekształceniu: 
$$Y(z)(1-1.6840z^{-1}+0.7050z^{-2}) = U(z)(0..388z^{-11}+0.0346z^{-12})$$
Skąd możemy bezpośrednio przejść do równania różnicowego: 
$$y(k) = 1.684(k-1) - 0.705(k-2) + 0.388u(k-11) + 0.0346u(k-12)$$


\section{Dobór regulatora PID metodą Zieglera-Nicholsa}
Transmitancja ciągłego regulatora PID wygląda następujaco: 
$$R(s) = k_r(1=\frac{1}{T_is}+T_ds)=K_p + K_i\frac{1}{s}+K_ds$$
Pierwszym etapem metody Zieglera-Nicholsa jest wyznaczenie wzmocnnienia krytyczneho $K$, dla którego sygnał wyjściowy utrzymuje się na poziomie oscylacji, wokół wartości zadanej, ze stałą amplitudą. Przy parametrach $K_i, K_d=0$ topniowo zwiększamy wartosc parametru $K_p$ aż do wystąpienia oscylacji niegasnących o stałej amplitudzie. Dla tak wyznaczonego wzmocnienia odczytujemy również okres oscylacji $T_{osc}$. Wysnaczone parametry wynoszą: \\

$K_{kr} = 0.6314$\\
\indent $T_{osc} = 17$\\

Posiadając wzmocnienie krytyczne i okres oscylacji możemy obliczyć pozostałe parametry na podstawie tabelki: 
\begin{table}[htp]
\centering
\caption{Tabela nastaw PID}
\label{my-label}
\begin{tabular}{|l|l|l|l|}
\hline
    & $K_p$       & $K_i $                & $K_d $              \\
\hline
P   & $0.5K_{kr} $ &                    &                   \\
\hline
PI   & $0.45K_{kr}$ & $\frac{1.2K_p}{T_{osc}}$ &                   \\
\hline
PID & $0.6K_{kr} $ & $\frac{2K_p}{T_{osc}}$   & $\frac{8K_p}{T_{osc}}$ \\
\hline
\end{tabular}
\end{table}
Wynoszą one :\\

$K_p = 0.37884$\\
\indent$K_i = 0.03714$\\
\indent$K_d = 0.14856$\\
\\

\begin{figure}[htp]
\centering
\includegraphics[scale=0.60]{2_1.png}
\caption{Odpowiedź układu na skok jednostkowy z wybranymi parametrami}
\label{}
\end{figure}

\subsection{Dyskretny regulator PID}
Transmitancja dyskretnego regulatora PID ma następującą postać: 
$$R(z) = \frac{r_2z^{-2} + r_1z^{-1} + r_0}{1-z^{-1}}$$
Korzystając z przekształcenia transformaty $Z$ otrzymujemy następujące równanie: 
$$R(z) = \frac{(K_p+K_d)+(K_iT_p-K_p-2K_d)z^{-1} + K_dz^{-2}}{1-z^{-1}}$$
Porównując z ogólną postacią transmitancji dyskretnej otrzymujemy wartości parametrów: \\

$r_2 = K_d$\\
\indent$r_1 = K_iT_p -K_p-2K_d$\\
\indent$r_0 = K_p + K_d$\\

\noindent Co po podstawieniu wyznaczonych nastaw daje: \\

$r_2 = 0.14856$\\ 
\indent $r_1 = -0.672246$\\
\indent $r_0 = 0.5274$\\

Okres próbokwania $T_p$ wynosi $0.5s$

\begin{figure}[htp]
\centering
\includegraphics[scale=0.60]{2_2.png}
\caption{Odpowiedź ukłądu na skok jednostkowy w przypadku regulatora dyskretnego}
\label{}
\end{figure}




\end{document}