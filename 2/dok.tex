% TeX encoding = utf8
% TeX spellcheck = pl_PL 
\documentclass[a4paper, 11pt]{article}
\usepackage[utf8]{inputenc}
\usepackage[polish]{babel}
\usepackage{polski}
\usepackage{float}
\usepackage{graphicx}
\usepackage{listings}
\usepackage{amsfonts}
\usepackage{geometry}
\usepackage{multicol}
\usepackage{latexsym}
\usepackage{enumerate}
\usepackage{hyperref}
\usepackage{wrapfig}
\usepackage{color} %red, green, blue, yellow, cyan, magenta, black, white
\definecolor{mygreen}{RGB}{28,172,0} % color values Red, Green, Blue
\definecolor{mylilas}{RGB}{170,55,241}

\author{Kamil Foryszewski}
\title{Sterowanie Procesami - projekt 2, zadanie 2.16}
\frenchspacing

\newgeometry{tmargin=2cm, bmargin=2cm, lmargin=2cm, rmargin=2cm}
\pagestyle{empty}


\begin{document}

\lstset{language=Matlab,%
    basicstyle=\color{red},
    breaklines=true,%
    morekeywords={matlab2tikz},
    keywordstyle=\color{blue},%
    morekeywords=[2]{1}, keywordstyle=[2]{\color{black}},
    identifierstyle=\color{black},%
    stringstyle=\color{mylilas},%
    commentstyle=\color{mygreen},%
    showstringspaces=false,
    numbers=right,%
    numberstyle={ \color{black}},% size of the numbers
    numbersep=5pt, % this defines how far the numbers are from the text
    emph=[1]{for,endfor,endwhile,endfunction,endif,break},emphstyle=[1]\color{blue}, %some words to emphasise
    emph=[2]{,.}, emphstyle=[2]\color{yellow},%
}

\maketitle
\tableofcontents

\section{Polecenie}
Obiekt regulacji jest opisany transmitancją: 
$$G(s) = \frac{K_0e^{-T_0s}}{(T_1 + 1)(T_2+1)}  = \frac{5e^{-T_0s}}{10.04s^2+7.02s +1}  $$
Gdzie: 
$K_0 = 3.5, T_0 = 5, T_1 = 2, T_2 = 5.02$

\section{Transmitancja dyskretna}
Wyznaczona transmitancja dyskretna ma następującą postać: 
$$G(z) = \frac{0.2539z^2-0.3048s+0.0858}{z^3-3.1287z^2-2.2119z-0.4493}$$
Została wyznaczona przy pomocy polecenia \emph{c2d} pakietu matlab. 
$$Gs = tf([2\quad 22\quad 48], [1\quad 8\quad -65\quad -504])$$
Na początku został utworzony model transmitancji ciągłej wykorzystany poźniej do wyznaczenia transmitancji dyskretnej. 
$$Gz = c2d(Gs,0.1,\ 'zoh')$$
Gdzie $0.1$ to okres próbkowania, \emph{'zoh'} oznacza ekstrapolator zerowego rzędu. 
\section{Równanie różnicowe}


\end{document}